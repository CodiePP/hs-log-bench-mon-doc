
% mu-benchmarks.tex

\subsection{Micro-benchmarks record observables}\label{sec:mubenchmarks}

Micro-benchmarks are recording observables that measure resource usage of the
whole program for a specific time. These measurements are then associated with
the subsystem that was observed at that time.
Caveat: if the executable under observation runs on a multiprocessor computer
where more than one parallel thread executes at the same time, it becomes
difficult to associate resource usage to a single function. Even more so, as
Haskell's thread do not map directly to operating system threads. So the
expressiveness of our approach is only valid statistically when a large number
of observables have been captured.

\subsubsection{Counters}

The framework provides access to the following O/S counters (defined in |ObservableInstance|) on \emph{Linux}:
\\
\begin{itemize}
\item monotonic clock (see |MonotonicClock|)
\item CPU or total time (\emph{/proc/\textless pid \textgreater/stat}) (see |ProcessStats|)
\item memory allocation (\emph{/proc/\textless pid \textgreater/statm}) (see |MemoryStats|)
\item network bytes received/sent (\emph{/proc/\textless pid \textgreater/net/netstat}) (see |NetStats|)
\item disk input/output (\emph{/proc/\textless pid \textgreater/io}) (see |IOStats|)
\end{itemize}

On all platforms, access is provided to the \emph{RTS} counters (see |GhcRtsStats|).

\subsubsection{Implementing micro-benchmarks}

In a micro-benchmark we capture operating system counters over an STM evaluation
or a function, before and afterwards. Then, we compute the difference between the
two and report all three measurements via a \emph{Trace} to the logging system.
Here we refer to the example that can be found in \hyperref[sec:examplecomplex]{complex example}.

\begin{wrapfigure}{r}{0.5\textwidth}
    \begin{center}\begin{scriptsize}\begin{verbatim}
    STM.bracketObserveIO trace "observeSTM" (stmAction args)
    \end{verbatim}\end{scriptsize}\end{center}
\end{wrapfigure}

The capturing of STM actions is defined in |Cardano.BM.Observer.STM| and the
function \emph{STM.bracketObserveIO} has type:
\begin{verbatim}
    bracketObserveIO
        :: Configuration
        -> Trace IO a
        -> Severity
        -> Text
        -> STM.STM t
        -> IO t
\end{verbatim}
It accepts a Trace to which it logs, adds a name to the context name and enters
this with a SubTrace, and finally the STM action which will be evaluated.
Because this evaluation can be retried, we cannot pass to it a Trace to which it
could log directly. A variant of this function |bracketObserveLogIO| also
captures log items in its result, which then are threaded through the Trace.
\\
Capturing observables for a function evaluation in \emph{IO}, the type of
\mbox{bracketObserveIO} (defined in |Cardano.BM.Observer.Monadic|) is:
\begin{verbatim}
    bracketObserveIO
        :: Configuration
        -> Trace IO a
        -> Severity
        -> Text
        -> IO t
        -> IO t
\end{verbatim}

It accepts a Trace to which it logs items, adds a name to the context name and
enters this with a SubTrace, and then the IO action which will be evaluated.

\begin{wrapfigure}{r}{0.5\textwidth}
    \vspace{-10pt}
    \begin{center}\begin{scriptsize}\begin{verbatim}
    bracketObserveIO trace "observeDownload" $ do
        license <- openURI "http://www.gnu.org/licenses/gpl.txt"
        case license of
            Right bs -> logInfo trace $ pack $ BS8.unpack bs
            Left e   -> logError trace $ "failed to download; error: " ++ (show e)
        threadDelay 50000  -- .05 second
        pure ()
    \end{verbatim}\end{scriptsize}\end{center}
  \end{wrapfigure}

Counters are evaluated before the evaluation and afterwards. We trace these as
log items |ObserveOpen| and |ObserveClose|, as well as the difference
with type |ObserveDiff|.

\subsubsection{Configuration of mu-benchmarks}

Observed STM actions or functions enter a new named context with a SubTrace.
Thus, they need a configuration of the behaviour of this SubTrace in the new
context. We can define this in the configuration for our example:
\begin{verbatim}
    CM.setSubTrace c "complex.observeDownload" (Just $ ObservableTrace [NetStats,IOStats])
\end{verbatim}

This enables the capturing of network and I/O stats from the operating system.
Other Observables are implemented in |Cardano.BM.Data.Observable|.
\\
Captured observables need to be routed to backends. In our example we configure:
\begin{verbatim}
    CM.setBackends c "complex.observeIO" (Just [AggregationBK])
\end{verbatim}
to direct observables from named context \emph{complex.observeIO} to the
Aggregation backend.
